% Created 2024-10-07 Mon 11:31
% Intended LaTeX compiler: xelatex
\documentclass[11pt]{article}
\usepackage[mathletters]{ucs}
\usepackage{graphicx}
\usepackage{longtable}
\usepackage{wrapfig}
\usepackage{rotating}
\usepackage[normalem]{ulem}
\usepackage{amsmath}
\usepackage{amssymb}
\usepackage{capt-of}
\usepackage{hyperref}
\usepackage[margin=0.5in]{geometry}
\author{Derek Prince}
\date{\textit{<2024-10-07 Mon>}}
\title{Thy Kingdom Come}
\hypersetup{
 pdfauthor={Derek Prince},
 pdftitle={Thy Kingdom Come},
 pdfkeywords={faith christianity},
 pdfsubject={},
 pdfcreator={Emacs 29.1.50 (Org mode 9.6.8)}, 
 pdflang={English}}
\begin{document}

\maketitle

\section{Thy Kingdom Come}
\label{sec:org62c2dc0}

Derek Prince preached this sermon some time in the 80s, I would guess.

\begin{description}
\item[{DerekPrince.com}] \href{https://www.derekprince.com/sermons/409}{derekprince.com: Thy Kingdom Come | Sermon | Derek Prince Ministries}

\item[{YouTube}] \href{https://www.youtube.com/watch?v=9L-ZM0g6yf0}{youtube.com: Thy Kingdom Come | Sermon by Derek Prince}
\end{description}

\subsection{Transcript}
\label{sec:org34b7ee3}

If I were to ask you here whom do you consider
the most influential kind of person on earth,
I suppose you might think of various different
answers.

Probably your mind would turn to political
leaders such as George Bush or Mikhail
Gorbachev.

Or, others, men of science.

Or, maybe even military commanders.

But I don't believe that they are the really
influential people.

I look in a completely different direction.

As I understand it, the most influential
people on earth today are those who know how
to get their prayers answered.

Because, they can release the omnipotence of
God into situations which goes far beyond
anything that the wisest or most powerful
human being can do.

So, I want to speak to you in this message on
certain requirements for getting our prayers
answered.

But first of all, I just want to read to you
three statements from the New Testament about
what prayer can do.

The first is found in \href{https://www.biblegateway.com/passage/?search=Matthew\%2021\%3A22\&version=ESV}{Matthew 21:22}.

Jesus Himself is speaking and He says:

\begin{quote}
"And all things whatever you ask in prayer, believing, you will receive."
\end{quote}

Notice, all things that you ask in prayer you will
receive.

There's no restrictions.

There's just one condition-believing.

The condition there is faith.

And then we turn to John's gospel, chapter 15
and verse 7.

And again we have the words of Jesus to His
disciples:

\begin{quote}
"If you abide in me, and my words
abide in you, you will ask what you desire and
it shall be done for you."
\end{quote}

Another version says "ask what you desire and it shall be done
for you." Notice there what you desire.

There's no other restriction, but the
condition is if my words abide in you.

In other words, if we pray according to God's
word.

And one final Scripture in the first epistle
of John, chapter 5, verses 14-15:

\begin{quote}
Now this is the confidence that we have in
Him [that is, in God], that, if we ask
anything according to His will, He hears us.

And if we know that He hears us in whatever
we ask, we know that we have the petitions
that we have asked of him.
\end{quote}

There again it says whatever we ask.

But again there's a condition attached, it's
according to His will.

So if you take those three passages together,
you'll find that the scope of prayer is
unlimited.

Nothing is excluded.

But, the conditions are threefold: faith,
according to God's word and according to God's
will.

And they go together because God's word
reveals God's will.

And we know that when we're praying according
to God's will we will get what we pray for.

So I want to deal in this message with what I
consider to be the basic requirements for
praying that kind of prayer, for making you a
person who can change the course of history.

I don't know whether you're aware, but in the
light of recent events in Eastern Europe, the
overthrow of the Iron Curtain, the release of
political liberty inside the Soviet Union.

In 1987 the church in Russian, which was
looking forward to its one thousandth
anniversary, set aside the month of January
for prayer and fasting.

I believe if you look for the real cause of
the dramatic political changes in Eastern
Europe and the Soviet Union, it's not in the
decision of politicians, it's in the prayers
of God's people.

I'd have to say, and I trust you'd agree with
me, that if that amount of authority has been
committed to us as believers in Jesus and in
the Bible, we are really foolish if we don't
make use of it.

I think we're negligent.

I think God is going to call us to account if
He's committed that kind of authority to us
and we don't appreciate it and make use of it.

Now, for the conditions that I want to speak
about I want to turn to the sermon on the
mount.

And particularly, to the 6th chapter.

I want to take the first part of what is
familiarly know to Christians around the earth
as the Lord's Prayer.

Jesus says in \href{https://www.biblegateway.com/passage/?search=Matthew\%206\%3A9\&version=ESV}{Matthew 6:9}:

\begin{quote}
"In this manner therefore pray\ldots{}"
\end{quote}

I don't believe He means that we've always
got to use precisely those words-although
they're beautiful words.

I believe He set forth a pattern of the way we
ought to pray.

And typically, it's very concise but very
complete.

I only want to take the first two verses,
verse 9 and verse 10.

I believe they contain the key to effective
praying.

I hope I can put this key in your hands and
you can walk out of this meeting with a key
that will unlock the omnipotence of God.

Let's just read those familiar words.

I'm reading from the New King James Version.

\begin{quote}
"Our Father in heaven, hallowed be your name,
your kingdom come, your will be done on earth
as it is in heaven."
\end{quote}

First of all, we address God as Father.

That makes all the difference.

We're not praying to some remote unknown
deity, some impersonal force; but we're
praying to a person who has made Himself our
Father through Jesus Christ.

You see, the mechanistic view of the universe
that it's just a series of material explosions
and causes leaves a person very lonely, lost
in the vastness of the universe which he
doesn't understand and can't control.

I always think of a friend of mine, a well
known Catholic Charismatic speaker.

If I gave you his name it would be know to
almost all of you-incidentally, he was
recently here in New Zealand so I'm giving a
lot away.

And, he told me years back he was in one of
the slums of the big cities of the United
States and it was late in the evening, it was
getting dark.

It was a cold, windy evening and the dust was
swirling up around him.

He was standing on the corner of the street
and he felt so lonely and so weak.

Then it came to him just to use one word and
he repeated again and again, "Father, Father,
Father," addressing it to God.

The more he repeated the word Father the
stronger and more secure he felt.

And just reinforcing that one relationship to
Almighty God as his Father completely changed
his whole outlook on his situation at that
moment.

For me, my background is in philosophy and I
studied for a good many years the various
theories about the origin of the universe.

I never could find one that satisfied me
intellectually.

Then I started to read the Bible in
desperation.

I thought at least it can't be any sillier
than some of the other theories I'd heard.

I didn't believe it was divinely inspired or
unique, I just said I'll treat it like any
other book, start at the beginning and read it
through to the end.

I did that when I went into the British Army
in 1940.

So, I took a Bible with me, planning to read
it through when I was in the army.

And of course, I had plenty of time because I
spent the next five and a half years in the
British Army-not by my choice.

I always recall the impact this made the first
night I was in the barrack room with twenty-
four other new recruits.

I didn't think anything about it, I just sat
down and opened up the Bible.

The other soldiers began to look at me and
they realized I was reading the Bible and an
uneasy hush fell on the whole barrack room.

I couldn't believe that one book would have
that much effect! And the thing was I baffled
them all because when I was not reading the
Bible I didn't live the least bit like people
who regularly read the Bible.

However, through reading the Bible I met the
author.

And once I met the author the book made the
most wonderful sense to me.

I found in the Bible the answers that I had
not found in philosophy.

I found a description of the beginning of
things that explained me to myself.

That was what really mattered.

When I read the description of creation of man
in Genesis chapters 1, 2 and 3, it explained
what was going on inside me.

See, my favorite philosopher whom I studied, I
read every word he ever wrote in the Greek
language, was Plato.

One of Plato's pictures was the human soul was
like a chariot drawn by two horses-one black
and one white.

The white horse was always trying to go
upwards, the black horse was always pulling
the chariot downwards.

And really that was so true to my experience.

I didn't get any further but when I read the
description of the creation of man I saw man
comes from two sources.

He's from the dust of the earth below but he's
from the breath of Almighty God above.

And there is in every one of us something of a
tension between what comes from above and what
comes from the earth.

But God shows us in His word how to resolve
that tension and bring our lives into harmony.

I can't go into that tonight but I just want
to testify that the Bible made the most
wonderful sense to me, it explained me to
myself.

I had a totally different view of the universe
from that time onwards.

When I met the God of the Bible I knew that
there wasn't some impersonal force, some big
bang that just brought the universe into
being; but I came to understand that there was
a Father and that the real power behind
everything is the love of God.

And one thing the Bible never explains is the
love of God.

It tells us God loves us, it never tells us
why God loves us.

The one unexplained fact in the universe is
the love of God.

We just have to receive it, we'll never
understand it.

Why God should love us passes our
comprehension but the good news is He really
does.

And so, Jesus says when you start to pray to
God, the first word you use is Father.

In the English translation it comes out "our
Father," but in the Greek "Father" comes first
and then "our." So the first thing you do when
you pray if you know God as Father through
Jesus Christ is approach Him as Father.

And then the word our is important because
most of us are extremely self centered and
when we pray we tend to pray, "Lord, bless me,
help me, heal me." Jesus reminds us, "You're
not the only child God has.

He's got a lot of other children and they're
all important to Him.

It's important to Him that you care for your
brothers and sisters." And then the next
phrase is "hallowed be thy name." That
expresses an attitude of reverence, of
worship.

And so after we've acknowledged God as Father
we need to adopt an attitude of reverence.

I have to say that in many sections of the
church today that's sadly lacking, the
approach of reverence to Almighty God.

God doesn't want us terrified but He does want
us reverent.

And something happens in our spirit when we
let that attitude of reverence express itself
in our prayers.

And then we come to the first two, if you
could call it, petitions.

And these are the ones I want to deal with.

\begin{quote}
"Your kingdom come\ldots{}"
\end{quote}

And then:

\begin{quote}
"\ldots{} your will be done on earth as it is in heaven."
\end{quote}

And notice we don't begin by praying
for what we need.

That follows:

\begin{quote}
"Give us this day our daily bread, and forgive us our trespasses\ldots{}"
\end{quote}

But that's not where we start.
We start with God's purposes, what's important
to God.

See, through the fall man was shut up in a
little prison called self.

Natural man is self-centered, his life focuses
on himself.

How can I get what I want?

Who's going to help me?

What do I get out of this?

That's a prison.

But through the new birth and through the
grace of God we can be released from that
prison of self-centeredness and enter into a
relationship with God where what God wants is
more important than what we want.

And when you pray that way you're beginning to
grow wings, you can move out of the natural
level.

Praying is not going to God with a shopping
list, did you know that?

After all, Jesus said your Father knows what
you need before you ask Him.

You don't have to tell God what you need.

What's important is that you get into such a
relationship with God that you know when you
do tell Him you're going to get it.

Establishing the relationship is much more
important than giving God a list of what you
need.

So, the first thing we are directed to say is
"Your kingdom come." That's tremendously
important because what we're doing is aligning
ourselves with God's purpose.

God's ultimate purpose in this age is very
simple in its essence-the details may be
complicated.

But the essential plan of God is simple: to
establish His kingdom on earth.

And notice Jesus says on earth.

That's God's first priority.

All through the history of this age from the
time that Jesus died and rose again until now,
God's priority has never changed.

Millions and millions of Christians pray the
Lord's Prayer and never realize what they're
praying for.

When we say "Your kingdom come," we are
aligning ourselves with what God wants done in
the earth.

You see, ultimately the only real, practical
solution to the needs of humanity is the
establishment of God's kingdom.

We hear a lot today about a social gospel,
meaning that we need to care for man's
physical and material needs.

I agree.

I believe all Christians should be concerned
with the physical and material needs of our
fellow human beings.

I believe that's the expression of love.

If you love people you'll be concerned about
their needs.

But, I don't believe that it's in our path
really to meet the needs of humanity as a
whole.

The church has been here nearly 2,000 years
and the needs in many cases are greater now
than they've ever been at any time in human
history.

40,000 children under the age of five die
every week on the earth today, mainly of
malnutrition, unsanitary conditions.

And yet, if all the money that was spent by
the nations on military armaments were made
available, it would be abundant to establish
hospitals, clinics, safe water supplies in
every nation on earth.

The problem is not the resources are not
available, the problem is that human greed and
fear and hate cause the resources to be
misdirected.

Now don't misunderstand me, I'm not preaching
pacifism, I'm just pointing out that the root
of the problem is in human nature.

And personally, I don't believe that man by
himself or the church by itself is ever going
to resolve the material and practical needs of
humanity.

It's only one thing that can do that.

What is that?

The establishment of God's kingdom on earth.

See, I claim to be a practical person.

I don't want to be just a visionary or a
dreamer.

I tell people many times the Holy Spirit is
the most practical person on the earth today.

If a thing is not practical it's not
spiritual.

I believe the establishment of God's kingdom
is the only practical solution to human need.

The people who preach what's called a social
gospel are presenting a dream.

Their motives may be good but to suggest that
by merely focusing on man's material needs we
can resolve them is not true.

There's only one hope for humanity.

Ruth and I travel widely and we've been to a
lot of places where people are desperately
poor and in need and in ignorance.

If you've never been outside this beautiful
nation you probably have a very faint picture
of the needs of humanity in many, many nations
across the earth.

And, they are not being met.

In many cases they are increasing.

Poverty, deprivation, hunger.

They're not less.

If anything, they're more.

I'm not a pessimist.

I believe there's a solution but I believe
it's God's solution.

And, it's a practical solution.

It's the establishment of Christ's kingdom on
earth.

I don't believe anything else is going to do
it.

God is a great realist and His love for
humanity causes Him to make priority number
one the meeting of the needs of humanity
through the establishment of Christ's kingdom
on earth.

I hope you understand me.

Now, we need to have a little clarity about
the way the kingdom is established.

This is a subject which is of great interest
to Christians at the present time.

First of all, the kingdom is defined in its
essential nature by Paul in one simple verse
in \href{https://www.biblegateway.com/passage/?search=Romans\%2014\%3A17\&version=ESV}{Romans 14:17}:

\begin{quote}
The kingdom of God is not food and drink; but righteousness, peace and
joy in the Holy Spirit.
\end{quote}

I want to point out two things.

First of all, righteousness comes first and
without true righteousness there will never be
true peace.

The world today is talking a great deal about
peace and many sections of the church are
praying for peace.

That's a good prayer but bear in mind without
righteousness peace will never come to this
earth.

God says twice in the prophet Isaiah there is
no peace to the wicked.

I meet many Christians who want peace and joy
but very often I find they've omitted the fact
that they only come as the results of
righteousness.

Righteousness is the first expression of the
kingdom.

And any attempt to achieve peace without
righteousness is doomed to frustration.

Personally, my understanding of Biblical
prophecy is that there will come an
antichrist, a satanically inspired ruler.

And he will promise peace and will seem for a
very brief moment to achieve it.

But Paul says in \href{https://www.biblegateway.com/passage/?search=1\%20Thessalonians\%205\%3A3\&version=ESV}{1 Thessalonians 5:3}:

\begin{quote}
When they say peace and security, then sudden
destruction comes upon them without warning.
\end{quote}

So the pursuit of peace bypassing
righteousness is foolishness.

So it's righteousness, peace and joy in the
Holy Spirit.

The only power that can impart those things in
their true nature is the Holy Spirit.

Somebody said the limits of the kingdom of God
are the limits of where the Holy Spirit is
present.

Where the Holy Spirit is not present you
cannot have the kingdom of God.

Now, the first way that the kingdom comes is
inwardly.

Jesus told the Pharisees of His day the
kingdom doesn't come by watching and waiting
for it externally.

He said the kingdom of God is within you or
among you.

There's no kingdom without a king.

So if we want the kingdom we have to welcome
the king.

And many of you here in this meeting know from
experience that when the King comes in He
bring His kingdom with Him.

But to want the kingdom apart from the King is
to deceive yourself.

So, there is an individual experience of the
kingdom of God that every true believer can
have who makes Jesus unreservedly king.

And that means displacing self from the throne
of your heart and placing Jesus on that
throne.

And when you do that then the kingdom of God
sets in.

Righteousness, peace and joy.

But Jesus also said the kingdom of God is
within you.

I believe there's a corporate expression of
the kingdom which is in the true community of
believers which is called the church.

Not some man-made institution but the
fellowship of those who've made Jesus king in
their own hearts and lives and relate to one
another on that basis.

I believe that it's the responsibility of the
church in any place to model the kingdom of
God, that by our attitudes and our
relationships and the way we live we challenge
the world with a glimpse of the kingdom.

And, people should be able to look at the
church and say, "So that's what the kingdom of
God is like." Righteousness, peace and joy in
the Holy Spirit.

I tell you that where the church demonstrates
these things, the hearts of men and women are
nearly always open to the truth of the gospel.

But if the world doesn't see the kingdom in
the church, why should the world believe our
message.

Probably they will not.

Let me just suggest one very practical and
important way in which we can model the
kingdom and it's a way that today is somewhat
controversial.

The fact of the matter is that today truth is
controversial.

Isaiah wrote about a time when truth has
fallen in the street and righteousness cannot
enter.

And we're not far from a time like this in
many parts of the human society.

But Paul said to Christian married couples:

\begin{quote}
"Husbands, love your wives as Christ loved the
church."
\end{quote}

And I tell husbands that's not a
recommendation, that's a commandment.

You are commanded to love your wife.

You don't have any options.

And furthermore, it will do you a lot of good
when you do it.

Some of you may have seen that book which has
rather a lengthy title, Husbands, Do
Yourselves a Favor, Love Your Wives.

It's true.

But the other side of it is:

\begin{quote}
"Wives, submit to your husbands as the church is submitted to Jesus Christ."
\end{quote}

If you put the first two statements together I
see it this way, that every Christian married
couple should be a prophetic message to the
world.

When the world looks at Christian married
couples the world should say, "I understand
that the way that man loves his wife is the
way Christ loved the church.

And, the way that woman relates to her husband
is the way the church relates to Christ."
Would you like to be prophetic?

I think all of us, in a way, would have an
ambition to be somewhat prophetic.

Well, a committed Christian couple can be
prophetic.

You can be a message to the world.

This is what the kingdom of God is like.

If there's one place that the kingdom should
be demonstrated first and foremost, it's in
the believer's family.

And if there's one place that Satan is
attacking today, it's the family.

Because, it was designed by God to represent
the kingdom and Satan wants to blur and
obscure and eliminate the message of the
kingdom.

He's afraid of the kingdom because wherever
the kingdom is established his power is come
to an end.

So there are two primary ways in which the
kingdom can come invisibly.

In the individual heart and life of the
believer and in the corporate fellowship of
the true church.

And, not least, in the Christian home.

But, that's not the ultimate.

The ultimate is the visible establishment of
God's kingdom on earth.

And the visible kingdom requires a visible
king.

Only when the King Himself has returned
visibly and in person can the true kingdom of
God be established on earth.

And personally, I have to say I feel it
presumptuous for the church to suggest that we
can do the job and finish it off without
Jesus.

The Bible says that we should be eagerly
longing for His appearing.

It says that in many different places.

I would like to ask you this evening, are you
eagerly longing for the appearing of Jesus?

If not, why not?

A friend of mine who is a preacher has a
rather droll way of expressing himself.

He said when Jesus comes back the church
should do something more than just say, "Nice
to have you back." Believe me, brothers and
sisters, things are going to happen on earth
between now and then that will make us
desperately anxious to see Him back.

God is going to arrange that.

So that's the primary purpose of God, the
establishment of His kingdom on earth visibly
with a visible king ruling over it.

And everything that God does is directed
toward that.

And until we make that our priority, we are
not really aligned with the will and purpose
of God.

See, that's why that's the first thing that we
are required to say after acknowledging God as
our Father.

We are required to align ourselves with God's
purpose.

I tell people this: Prayer is not a way for
you to get God to do what you want.

A lot of Christians think it is.

It may work out that way but that's not its
purpose.

Prayer is a way for you to become an
instrument for God to do what He wants.

All right?

I think I'm going to say that again.

Prayer is not a way for you to get God to do
what you want.

Prayer is a way for you to become an
instrument for God to do what He wants.

And when you become aligned with God's purpose
you're going to pray prayers that are
irresistible.

There'll be no power, human or satanic, that
will be able to resist the outworking of your
prayer.

And then Jesus said one more thing:

\begin{quote}
"Your will be done on earth as it is in heaven."
\end{quote}

Do you believe that's possible?

Do you really believe that when we pray
rightly in any given situation God's will can
be done as completely here on earth as it's
being done in heaven?

I do.

It doesn't mean everything is perfect on earth
but it means in any given situation God's
purpose and solution can be perfectly worked
out.

Do you believe that?

You see, you'll pray differently if you
convince yourself of that.

But, if you say to God, "Your will be done,"
do you know what you're saying?

Not my will.

So that's the second essential requirement
that you renounce your own will wherever it
conflicts with God's will.

And I want to tell you this, God's will is
best.

Most of us have let the devil make us afraid
of God's will.

"Oh, if I embrace the will of God it means
suffering.

It means denial, I'm going to have to give
things up." It could happen that way but I
read in \href{https://www.biblegateway.com/passage/?search=Revelation\%204\%3A11\&version=ESV}{Revelation 4:11}:

\begin{quote}
All things were created for your will [or according to your will].
\end{quote}

And I've pondered on that verse many times,
partly because there's a song that's put out
by Scripture in Song on their first major
album.

They did \href{https://www.biblegateway.com/passage/?search=Revelation\%204\%3A11\&version=ESV}{Revelation 4:11} as a song.

And it particularly blessed me and so I
pondered on that verse many times.

I was thinking it over sometime back and I
suddenly realized that there could not be
anything better than God's will.

God's will is the best way for anything to be
at any time.

So, don't be afraid of embracing God's will.

But, you do it without knowing what it'll
involve.

When Ruth and I were preparing for these
meetings, theoretically we were resting in
Hawaii but actually we were battling the
forces of Satan.

We came to a point where both of us, I think
we were on the floor, said, "Lord, we embrace
your will without any reservation whatever.

Whatever you will, we embrace." And I think
God was kind of squeezing us, putting pressure
on us, to bring us to that place of total
surrender and embracing of the will of God.

And there comes a relief when you do that.

You don't know exactly what you're praying for
but remember, you've got a Father who loves
you, who's omnipotent, who always wants the
best for you.

It's foolish to press your will against God's
will.

His will is best.

As I look back over nearly 50 that I've walked
with the Lord, again and again I thank God for
the times that He didn't let me have my way.

I can see situation after situation, if I'd
done things the way I wanted, it would have
been disastrous.

I thank Him.

I thank God for the prayers He's answered, I
also thank Him for the prayers He didn't
answer.

He knows what is best.

So there are the two, I would say, attitudes,
the two ways you have to align yourself to
become an effective pray-er, to be able to
pray the kind of prayers that will change
nations, change situations, change families.

I look back on my own walk with the Lord and I
can say to the glory of God I can see several
points in the history of the last 50 years
where history was changed by my prayers.

I have a book, I don't know whether it's
available, called Shaping History Through
Prayer and Fasting.

And in it I give about four or five specific
examples of history changed when I prayed the
will of God.

I was saved in Britain in the army in 1941 and
just a few weeks later I was sent overseas
with my unit to North Africa.

I spent the next four and a half years in the
Middle East.

And during that time my unit took part in the
longest retreat recorded in the history of the
British Army, 750 miles, from a place called
El Agala in Tripoli, to the well known place
in Egypt, El Alamein.

I can tell you it's somewhat depressing to be
retreating continually for 750 miles,
particularly in a very barren and unattractive
desert.

And there I was, newly converted, I hadn't had
any opportunity to attend church.

All I had was the Bible and the Holy Spirit.

I thought to myself I ought to be able to pray
about this situation intelligently.

I knew I didn't know what to pray.

So I said in my naive way, "Lord, show me how
you want me to pray." And the Lord gave me a
specific answer which was this: "Lord, give us
leaders such that it will be to your glory to
give us victory through them." I was less than
a year old in the Lord when I prayed that
prayer.

I prayed it consistently.

And then the command of what was then the
Eighth Army was changed and Montgomery took
over by a series of strange accidents which I
can't go into in detail.

He was not destined to be commander.

And Montgomery at that time-I read an article
just recently in a British newspaper on the
100th anniversary of Montgomery's birth in
which it says that no British general in human
history has ever conducted a more brilliant
campaign than Montgomery conducted at that
time in North Africa.

Do you know why?

Shall I tell you why?

Because I prayed.

Do you believe that?

Can you believe that God will do things for
you.

Now I know there were other Christians in
Britain praying.

And after the battle of El Alamein, which was
the turning point of the war in that area, had
been fought, I was with my unit's lorry in the
desert somewhere and there was a little
portable radio on the tail board of the lorry.

There was a newscaster giving an eyewitness
account of the scene at Montgomery's
headquarters the night before the battle of El
Alamein was fought.

And he reported how General Montgomery came
out and called together his officers and his
men and said, "Let us ask the Lord, mighty in
battle, to give us the victory." And when I
heard that, I don't know whether you know what
heaven's electricity is like, some of you do,
but it went right through me from the crown of
my head to the soles of my feet.

God spoke to me inaudibly, "That's the answer
to your prayer." See, to be able to pray like
that is worth more than all the fortune in
this world.

A person who can pray like that is more
influential than the general who wins the
victory, the government that controls the
general.

Now I haven't always prayed that kind of
prayer.

Sometimes I've got bogged down in my own
little petty concerns and my limitations and I
got praying, "Give me, help me, bless me, heal
me." There's nothing wrong in asking God to
heal you but it won't really produce the
divine result until your whole attitude and
motivation is aligned with God's purpose in
the earth.

See, God's not going to change His purposes.

If God and I are out of harmony, guess who's
going to change?

There's only one option.

I have to change.

And living out of harmony with God, especially
if you're a Spirit baptized believer, is
painful.

How many of you would agree to that?

Don't put your hands up! Sit on your hands at
that point.

But how can we be in harmony?

The answer is align with God's purpose.

And the first two key steps are there at the
beginning of the Lord's Prayer, "Thy kingdom
come." Lord, I'm not praying for my own
interests or what I think is important, I'm
praying for the thing that's important in your
eyes, the thing that really is the solution-
the coming of your kingdom.

And then, Lord, I submit my will to you
personally.

Your will be done.

And where your will and my will clashes, Lord,
I say not my will be done.

I will tell you that there will always come a
point in the life of every believer where we
have to say not my will but yours be done.

And it's often extremely painful at the time.

But the end is always blessed.

I see some of you smiling and nodding your
heads.

You've found it for yourself.

And then in \href{https://www.biblegateway.com/passage/?search=Matthew\%206\%3A33\&version=ESV}{Matthew 6:33}, Jesus gives one more
direction which is the outworking of what
we've already discussed.

How many of you know what \href{https://www.biblegateway.com/passage/?search=Matthew\%206\%3A33\&version=ESV}{Matthew 6:33} says?

\begin{quote}
"But seek ye first the kingdom of God and His
righteousness, and all these things will be
added to you."
\end{quote}

Get your priorities right and
you really don't have to spend a lot of time
praying for your personal needs.

God will take care of your personal needs if
you align with His priorities.

As I say, I've been a Christian almost 50
years.

It's hard to believe.

I've been responsible for this ministry which
really is a worldwide ministry for at least 10
years now.

We marvel continually at how God supplies our
need.

We don't have any big resources, we don't have
any millionaires that lavish their fortune on
us.

Most of the people that support this ministry
are comparatively humble people.

And most of them ask for really no
recognition.

And yet I'd have to say-and here's the members
of our staff here-God has never failed to
supply our needs once.

And continually we have to reach out in faith.

We don't wait until the resources are there
and say now we can do it.

We say if this is what God wants us to do,
we'll do it and trust Him for the resources.

Years ago I was preaching to a group of young
people in the States on how to arrange your
life in line with God.

I said-it was a risky thing to say but I said,
when I go into a store I don't ask can I
afford this, I ask is this what God wants me
to have.

Because, if God wants me to have it I can
afford it.

Well, about six of those young people went out
after the service was over to a restaurant
across the road to get a little supper.

I heard this from them personally.

And they began to look at the menu and say
what can we afford?

And one of them said let's do what the
preacher said and just decide what does God
want us to have and order it.

I mean, I'm glad I wasn't there because I
would have been trembling.

So they did that and when they wanted to pay
the waiter said, "That gentleman over there
has just paid your bill!" But listen, don't do
it unless you have the faith.

Do you know what will happen?

You'll spend your evening washing greasy
dishes.

Now, one more vital question and we close this
message, which is how can I discover God's
will?

And I want to turn to one of my favorite
passages, Romans 12, and very briefly outline
the steps to discovering God's will.

As I understand it, they're all here in \href{https://www.biblegateway.com/passage/?search=Romans\%2012\%3A1-8\&version=ESV}{Romans 12:1-8}.

We don't have long but I'll go through it
briefly.

Paul begins with a therefore in verse 1, and
many of you have heard me say when you find a
therefore in the Bible, you want to find out
what it's there for.

And this therefore is because of the previous
11 chapters of Romans in which Paul has
outlined the whole message of God's mercy and
grace.

And he says In the light of that, what should
we do, how shall we respond?

And the answer is:

\begin{quote}
Present your body a living sacrifice, holy, acceptable to God\ldots{}
\end{quote}

That always blesses me, the Bible is so down
to earth.

A lot of us would expect something super-
spiritual, you know.

After all this glorious unfolding of the grace
of God, "God, what do you want?" "I want your
body." You see, when He gets our body, He gets
the contents.

He is so wise.

So He says, "Present your body to me a living
sacrifice." Why does it say a living
sacrifice?

Because Paul's contrasting it with the Old
Testament sacrifices which were first killed
and placed on the altar.

But Paul says don't kill your body and place
it on the altar, place a living body on the
altar.

When a sacrifice was placed on the altar it no
longer belonged to the person who offered it,
it belonged to God.

So God says place your body on my altar as a
living sacrifice and from now on you don't own
it.

I own it.

You don't make the decisions as to what will
happen to your body, I make them.

You don't decide where you're going to go, you
don't decide what you're going to eat, you
don't decide what you're going to wear, those
are my decisions, my responsibility, I take
full responsibility for your body.

And I'll tell you, He's much better able to
look after our bodies than we can.

You can't lose by putting God in charge of
your body.

But it has to be a genuine surrender.

And then the next verse says:

\begin{quote}
Do not be conformed to this world [or this age], but be
transformed by the renewing of your mind, that
you may prove what is the good and acceptable
and perfect will of God.
\end{quote}

You see, to discover God's will you have to
have a change in your thinking.

Your mind has to be renewed.

And God can do that but He will not do it
until He has your body.

He says present your body, then I'll renew
your mind.

And when your mind is renewed you can discover
the will of God.

You can't discover it with your old unrenewed
mind.

Lots of people get saved and I suppose they
get to heaven at the end but they never
discover God's will in this life because their
mind is never renewed.

And then Paul says in the next verse (\href{https://www.biblegateway.com/passage/?search=Romans\%2012\%3A3\&version=ESV}{Romans 12:3}):

\begin{quote}
I say through the grace given to me to everyone who
is among you, not to think of himself more
highly than he ought to think; but to think
soberly, as God has dealt to each one a
measure of faith.
\end{quote}

So the renewed mind is not proud, it's not
arrogant, it's not self asserting.

It is humble, it is sober, it's realistic.

The first day you walk into the bank to work,
don't expect to sit in the manager's chair.

See, that's unrealistic.

You come into the kingdom, don't expect to be
an apostle on day one.

Be willing to be an office boy, empty the
wastepaper baskets.

You know, in the spiritual life, you know the
way up?

It's down, that's right.

The lower down you go the higher up you end.

And then Paul says in effect, and I'm not
going to read the verse but he says you're not
going to make it on your own.

You're going to have to be part of the body of
Christ.

And God has given you a measure of faith, a
proportion of faith, suitable to your place in
the body.

So, you have to find your place in the body
and when you find your place you'll discover
that you have the faith that you need for that
place and for that function.

You see, my hand works wonderfully well as a
hand but if I try to walk on my hands I get
into trouble.

Because, my hand is designed to be a hand and
not a foot.

Lots of Christians are feet trying to be hands
or noses trying to be ears.

And if you have a continual struggle for faith
in your Christian walk it's almost a sure
guarantee that you're trying to be something
God didn't design you to be.

Because, basically the life of faith has tests
and problems but it flows.

It's not a continual struggle.

And when you find your place in the body your
allotted proportion of faith that God has
given you will make you successful in that
place.

And then finally Paul says, in closing this
little simple outline, when you're in your
place in the body, God will give you the gifts
you need for that place.

A lot of people just are interested in
spiritual gifts, and I agree, they're
exciting.

But they are not to be sought in detachment
from the body.

Because until you know your place in the body
you don't know what gifts you'll need.

My experience has been that when I get in the
right place I have the right gifts.

I remember when God thrust me into the
ministry of deliverance, helping people to be
free from demons, a certain friend of mine
brought his sister, a married woman, for
deliverance, to my first wife and me in a
hotel somewhere in Colorado.

And she sat there and she was a picture of
misery.

She obviously had problems.

I looked at her and I opened my mouth and I
heard myself say, "You need to be delivered
from\ldots{}" and I named about eight demons.

And then I thought to myself how did I know
that?

And then I realized God had given me the word
of knowledge.

Why?

As an ornament?

No.

Because I needed it to be effective in the
place that He put me.

But remember that the key, the initial move,
is to present your body a living sacrifice to
God.

That's a definite experience.

And as I close this message I want to
challenge you.

Have you ever really handed over the control
of your body to the Lord Jesus Christ?

Have you really said, "God, it's yours.

It's at your disposal, do with it what you
want." And if you haven't, there's no better
time than now to make that decision.

Shall we pray together just for a few moments?

I don't want to pressure anybody but it would
be unfair to close this message without giving
you an opportunity to respond.

If you've heard tonight something about what
God requires of you, a condition you haven't
fulfilled and you want to fulfill it, you want
to align yourself with God's purposes and
God's priorities, and you say, "God, here I
am.

Here's my body, I place it on your altar." If
you feel God prompting you to make that
decision, we want to help you.

Very, very simply, just ask you to do one
thing to indicate your decision, just stand to
your feet where you are in your place as an
act of surrender to God.

Say, "I take my hands off and I hand myself
over to God and His will." You want to make
that decision, stand to your feet wherever you
are.

Bless you.

Now this is a very serious decision and I
don't want to pressure you into making a
decision that you'll go back on.

God doesn't expect you to be perfect from this
moment onwards but He does expect you to be
sincere and wholehearted.

So, if you really have decided that this is
the night that you're going to put your body
on God's altar, I suggest that you say this
simple prayer out loud after me.

You're not praying to me, I'm just giving you
the words with which you can approach God.

It'll be very brief and very simple, it
doesn't have to be a long prayer.

You say these words:

\begin{quote}
"Lord Jesus Christ, I thank you that on the cross you died in my
place to save me from my sin and to make me a child of God.

And in response to your mercy, Lord, I now
present my body to you.

I lay it upon the altar of your service as a
living sacrifice.

From tonight onwards it belongs to you, Lord,
and not to me.

I thank you you receive this sacrifice because
I offer it to you through Jesus. \textbf{Amen.}"
\end{quote}

Now just continue a few moments just quietly thanking God.
\end{document}