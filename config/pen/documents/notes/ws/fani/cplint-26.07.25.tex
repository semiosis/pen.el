% Created 2025-07-26 Sat 16:38
% Intended LaTeX compiler: pdflatex
\documentclass[11pt]{article}
\usepackage[utf8]{inputenc}
\usepackage[T1]{fontenc}
\usepackage{graphicx}
\usepackage{longtable}
\usepackage{wrapfig}
\usepackage{rotating}
\usepackage[normalem]{ulem}
\usepackage{amsmath}
\usepackage{amssymb}
\usepackage{capt-of}
\usepackage{hyperref}
\usepackage[margin=0.5in]{geometry}
\author{Fani}
\date{\textit{<2024-06-17 Mon>}}
\title{cplint learning Day 1}
\hypersetup{
 pdfauthor={Fani},
 pdftitle={cplint learning Day 1},
 pdfkeywords={cplint prolog},
 pdfsubject={},
 pdfcreator={Emacs 29.4.50 (Org mode 9.6.15)}, 
 pdflang={English}}
\begin{document}

\maketitle

\section{Installation}
\label{sec:orga9c131b}
\subsection{1) Install SWI prolog}
\label{sec:orgf91c2ae}
\url{https://www.swi-prolog.org/download/stable}

\subsection{2) Install cplint}
\label{sec:org21b7812}
It is explained here:
\begin{itemize}
\item \url{http://friguzzi.github.io/cplint/\_build/html/index.html}
\end{itemize}

I also had to make sure this runs, after installing cplint.

\begin{verbatim}
pack_rebuild(bddem).
\end{verbatim}

\section{cplint Documentation}
\label{sec:org4017117}
\url{http://friguzzi.github.io/cplint/\_build/html/index.html}

It looks like it's possible to use R with cplint:

\begin{verbatim}
 1  Using R¶
 2  
 3  You have to load library cplint_r (a SWI-Prolog pack) with
 4  
 5  :- use_module(library(cplint_r)).
 6  
 7  Then you can use predicates
 8  
 9  bar_r/1
10  bar_r/2
11  argbar_r/1
12  
13  that work as their C3.js counterpart but do not return the graph as an
14  argument as the graph is printed with a different mechanism.
15  
16  You also have
17  
18  histogram_r(+List:list,+Options:list) is det
\end{verbatim}

\section{Learning:}
\label{sec:org01342f2}
\href{https://www.youtube.com/watch?v=SykxWpFwMGs\&pp=ygUSZGVyZWsgYmFuYXMgcHJvbG9n}{youtube.com: Prolog Tutorial}

\begin{verbatim}
  1  % Prolog programs are a collection of Facts, and Rules that we can
  2  % Query.
  3  
  4  % Prolog focuses on describing facts and relationships about problems
  5  % rather then on creating a series of steps to solve that problem.
  6  
  7  % These Facts and Rules are stored in a file called a Database
  8  % or Knowledge Base
  9  
 10  % You load a knowledge base like this [knowledge]. or this
 11  % consult('knowledge.pl').
 12  % halt. exits the prolog system
 13  % listing. Displays the contents of the database
 14  % All these commands are called predicates
 15  
 16  % ---------- INTRODUCTION ----------
 17  % write prints text between quotes to the screen
 18  % nl stands for new line and \'s allows you to use quotes
 19  % write('Hello World'),nl,write('Let\'s Program').
 20  
 21  % This is a fact where loves is a predicate and romeo and
 22  % juliet are atoms (constants) and loves arguments
 23  loves(romeo, juliet).
 24  
 25  % This is a rule where :- (if) says if the item on the right is
 26  % true, then so is the item on the left
 27  loves(juliet, romeo) :- loves(romeo, juliet).
 28  
 29  % Evaluating whether the goal was met in the terminal
 30  % loves(juliet, romeo). = yes
 31  
 32  % Facts and Rules are called clauses
 33  
 34  % A Variable is an object we can't name at the time of execution
 35  % Variables are uppercase while atoms are lowercase
 36  % loves(romeo, X). = X = juliet
 37  
 38  % ---------- FACTS ----------
 39  % Write the relationship first followed by the objects between
 40  % parenthese followed by a dot
 41  
 42  % albert, male, female are atom constants that must begin with a
 43  % lowercase letter unless they are between single quotes
 44  % An atom can contain letters, numbers, +, -, _, *, /, <, >, :, ., ~, &
 45  % AN ATOM CANNOT START WITH _
 46  
 47  % The name before parenthese is called the predicate
 48  % The names in parenthese are called arguments
 49  
 50  % Let's define information about the people above
 51  
 52  male(albert).
 53  male(bob).
 54  male(bill).
 55  male(carl).
 56  male(charlie).
 57  male(dan).
 58  male(edward).
 59  
 60  female(alice).
 61  female(betsy).
 62  female(diana).
 63  
 64  % We can find out if alice is a woman with
 65  % female(alice). = yes
 66  % listing(male). = list all clauses defining the predicate male
 67  % male(X), female(Y). = Show all combinations of male and female
 68  
 69  % ---------- RULES ----------
 70  % Rules are used when you want to say that a fact depends on a group of facts
 71  
 72  % NOTE : You'll get the discontiguous predicate warning if you
 73  % don't keep your predicates together
 74  
 75  happy(albert).
 76  happy(alice).
 77  happy(bob).
 78  happy(bill).
 79  with_albert(alice).
 80  
 81  % We can define the Fact that when Bob is happy he runs
 82  % :- stands for if
 83  runs(albert) :- happy(albert).
 84  % runs(albert). = yes
 85  
 86  % We can check if 2 conditions are true by putting a comma (and)
 87  % between questions (CONJUCTIONS)
 88  dances(alice) :-
 89    happy(alice),
 90    with_albert(alice).
 91  
 92  % We can define predicates to keep commands brief
 93  does_alice_dance :- dances(alice),
 94         write('When Alice is happy and with Albert she dances').
 95  % Just type does_alice_dance. in the terminal
 96  
 97  % Both rules must be true to get a yes result
 98  swims(bob) :-
 99    happy(bob),
100    near_water(bob).
101  % swims(bob). = no
102  
103  % We can create 2 instances and if either comes back true the result
104  % will be yes
105  swims(bill) :-
106    happy(bill).
107  
108  swims(bill) :-
109    near_water(bill).
110  % swims(bill). = yes
111  
112  % ---------- VARIABLES ----------
113  % A variable is an object we are unable to name when writing a program.
114  % An instantiated variable is one that stands for an object.
115  % A variable begins with an uppercase letter or _ and can contain
116  % the same symbols as atoms.
117  % The same variable name used in 2 different questions represents 2
118  % completely different variables.
119  
120  % An uninstantiated variable can be used to search for any match.
121  
122  % Return all females (Type ; to cycle through them)
123  % female(X). X = alice X = betsy X = diana
124  
125  parent(albert, bob).
126  parent(albert, betsy).
127  parent(albert, bill).
128  
129  parent(alice, bob).
130  parent(alice, betsy).
131  parent(alice, bill).
132  
133  parent(bob, carl).
134  parent(bob, charlie).
135  
136  % When you are cycling through the results the no at the end signals
137  % that there are no more results
138  % parent(X, bob). X = albert, X = alice
139  
140  % parent(X, bob), dances(X). X = alice
141  
142  % Who is Bobs parent? Does he have parents?
143  % parent(Y, carl), parent(X, Y). = X = albert, Y = bob, X = alice
144  % Y = bob
145  
146  % Find Alberts grandchildren
147  % Is Albert a father? Does his children have any children?
148  % parent(albert, X), parent(X, Y). = X = bob, Y = carl, X = bob,
149  % Y = charlie
150  
151  % Use custom predicate for multiple results
152  get_grandchild :- parent(albert, X), parent(X, Y),
153                write('Alberts grandchild is '),
154                write(Y), nl.
155  
156  % Do Carl and Charlie share a parent
157  % Who is Carls parent? Is this same X a parent of Charlie
158  % parent(X, carl), parent(X, charlie). = X = bob
159  
160  % Use format to get the results
161  % ~w represents where to put each value in the list at the end
162  % ~n is a newline
163  % ~s is used to input strings
164  get_grandparent :- parent(X, carl),
165                  parent(X, charlie),
166                  format('~w ~s grandparent~n', [X, "is the"]).
167  
168  % Does Carl have an Uncle?
169  % Who is Carls parent? Who is Carls fathers brother?
170  brother(bob, bill).
171  % parent(X, carl), brother(X, Y). = X = bob, Y = bill
172  
173  % Demonstrate axioms and derived facts
174  % We can also use variables in the database
175  % If you get the singleton warning, that means you defined a variable
176  % that you didn't do anything with. (This is ok sometimes)
177  grand_parent(X, Y) :-
178    parent(Z, X),
179    parent(Y, Z).
180  % grand_parent(carl, A). = A = albert, A = alice
181  
182  % X blushes if X is human
183  blushes(X) :- human(X).
184  human(derek).
185  
186  % If we say one thing is true when somehing else is true, we can also
187  % find that match if we only assign one thing to be true here.
188  % blushes(derek). = yes
189  
190  % Another example on cause and effect
191  stabs(tybalt,mercutio,sword).
192  hates(romeo, X) :- stabs(X, mercutio, sword).
193  % hates(romeo, X). = X = tybalt
194  
195  % We can use _ (anonymous variable) if we won't use the variable
196  % more than once
197  % The value of an anonymous var is not output
198  % Check if any males exist in the database : male(_). = yes
199  
200  % ---------- WHERE IS IF? ----------
201  % You can use a type of case statement instead
202  
203  what_grade(5) :-
204    write('Go to kindergarten').
205  what_grade(6) :-
206    write('Go to first grade').
207  what_grade(Other) :-
208    Grade is Other - 5,
209    format('Go to grade ~w', [Grade]).
210  
211  % ---------- COMPLEX TERMS / STRUCTURES ----------
212  % A Structure is an object made up from many other objects (components)
213  % Structures allow us to add context about what an object is to avoid
214  % confusion. has(albert,olive) Does Albert have a pet named Olive?
215  % Does Albert have the food named Olive?
216  
217  % Structures have a functor followed by a list of arguments
218  % The number of arguments a Structure has is its arity
219  % female(alice). has an arity of one
220  
221  % Albert owns a pet cat named Olive
222  % This is a recursive definition
223  
224  owns(albert, pet(cat, olive)).
225  
226  % owns(albert, pet(cat, X)). : X = olive
227  
228  customer(tom, smith, 20.55).
229  customer(sally, smith, 120.55).
230  
231  % An anonymous variable is used when we don't want a value returned
232  % Is there a customer named sally and what is her balance
233  % customer(sally,_,Bal).
234  
235  % tab puts the defined number of spaces on the screen
236  % ~2f says we want a float with 2 decimals
237  get_cust_bal(FName, LName) :- customer(FName, LName, Bal),
238    write(FName), tab(1),
239    format('~w owes us $~2f ~n', [LName, Bal]).
240  
241  % Use a complex term to define what it means to be a vertical
242  % versus a horizontal line
243  vertical(line(point(X, Y), point(X, Y2))).
244  horizontal(line(point(X, Y), point(X2, Y))).
245  
246  % vertical(line(point(5, 10), point(5, 20))). = yes
247  % horizontal(line(point(10, 20), point(30, 20))).
248  
249  % We can also ask what the value of a point should be to be vertical
250  % vertical(line(point(5, 10), point(X, 20))). = X = 5
251  
252  % We could also ask for the X and Y points
253  % vertical(line(point(5, 10), X)). = X = point(5,_)
254  
255  % ---------- COMPARISON ----------
256  % alice = alice. = yes
257  % 'alice' = alice. = yes (Prolog considers these to be the same)
258  % \+ (alice = albert). = yes (How to check for not equal)
259  
260  % 3 > 15. = no
261  % 3 >= 15. = no
262  % 3 =< 15. = yes
263  
264  % W = alice. = yes
265  % This says that we can assign the value of alice to W and not that
266  % W is equal to alice
267  
268  % Rand1 = Rand2. = yes
269  % This says that any variable can be assigned anything and one of
270  % those things is another variable
271  
272  % If variables can be matched up between 2 complex terms and the
273  % functors are equal then the complex terms are equal
274  % rich(money, X) = rich(Y, no_debt).
275  
276  % ---------- TRACE ----------
277  % Using trace we can see how Prolog evaluates queries one at a time
278  
279  warm_blooded(penguin).
280  warm_blooded(human).
281  
282  produce_milk(penguin).
283  produce_milk(human).
284  
285  have_feathers(penguin).
286  have_hair(human).
287  
288  mammal(X) :-
289    warm_blooded(X),
290    produce_milk(X),
291    have_hair(X).
292  
293  
294  % trace.
295  % mammal(human).
296  %       1    1  Call: mammal(human) ?
297  %       2    2  Call: warm_blooded(human) ?
298  %       2    2  Exit: warm_blooded(human) ?
299  %       3    2  Call: produce_milk(human) ?
300  %       3    2  Exit: produce_milk(human) ?
301  %       4    2  Call: have_hair(human) ?
302  %       4    2  Exit: have_hair(human) ?
303  %       1    1  Exit: mammal(human) ?
304  % yes
305  
306  % mammal(penguin).
307  %       1    1  Call: mammal(penguin) ?
308  %       2    2  Call: warm_blooded(penguin) ?
309  %       2    2  Exit: warm_blooded(penguin) ?
310  %       3    2  Call: produce_milk(penguin) ?
311  %       3    2  Exit: produce_milk(penguin) ?
312  %       4    2  Call: have_hair(penguin) ?
313  %       4    2  Fail: have_hair(penguin) ?
314  %       1    1  Fail: mammal(penguin) ?
315  % no
316  %
317  % notrace. Turns off trace
318  
319  % Output what ever matches the clauses
320  % warm_blooded(X), produce_milk(X), write(X),nl.
321  
322  % ---------- RECURSION ----------
323  
324  /*
325  parent(albert, bob).
326  parent(albert, betsy).
327  parent(albert, bill).
328  
329  parent(alice, bob).
330  parent(alice, betsy).
331  parent(alice, bill).
332  
333  parent(bob, carl).
334  parent(bob, charlie).
335  */
336  
337  % Works for exact matches
338  related(X, Y) :- parent(X, Y).
339  % related(albert, bob). = true
340  
341  % Cycles through possible results until related returns a true
342  related(X, Y) :-
343    parent(X, Z),
344    related(Z, Y).
345  
346  % related(albert,carl). = true
347  
348  % 1. parent(albert, Z). = true = Z = bob, betsy, bill
349  % 2. related(Z, carl). = true when Z = bob
350  
351  % ---------- MATH ----------
352  % Prolog provides 'is' to evaluate mathematical expressions
353  % X is 2 + 2. = X = 4
354  
355  % You can use parenthese
356  % X is 3 + (2 * 10). =  X = 23
357  
358  % You can also make comparisons
359  % 50 > 30. = yes
360  % (3*10) >= (50/2). = yes
361  % \+ (3 = 10). = yes (How to check for not equal)
362  % 5+4 =:= 4+5. = yes (Check for equality between expressions)
363  % 5+4 =\= 4+5. = yes (Check for non-equality between expressions)
364  % 5 > 10 ; 10 < 100. (Checks if 1 OR the other is true)
365  
366  % X is mod(7,2). = X = 1 (Modulus)
367  
368  double_digit(X,Y) :- Y is X*2.
369  % double_digit(4,Y). = Y = 8
370  % Take the 1st argument, multiply it times 2 and return it as the
371  % 2nd argument
372  
373  % Get random value between 0 and 10
374  % random(0,10,X).
375  
376  % Get all values between 0 and 10
377  % between(0,10,X).
378  
379  % Add 1 and assign it to X
380  % succ(2,X).
381  
382  % Get absolute value of -8
383  % X is abs(-8).
384  
385  % Get largest value
386  % X is max(10,5).
387  
388  % Get smallest value
389  % X is min(10,5).
390  
391  % Round a value
392  % X is round(10.56).
393  
394  % Convert float to integer
395  % X is truncate(10.56).
396  
397  % Round down
398  % X is floor(10.56).
399  
400  % Round up
401  % X is ceiling(10.56).
402  
403  % 2^3
404  % X is 2** 3.
405  
406  % Check if a number is even
407  % 10//2 = 5 (is 10 = 2 * 5)
408  is_even(X) :- Y is X//2, X =:= 2 * Y.
409  
410  % sqrt, sin, cos, tan, asin, acos, atan, atan2, sinh, cosh, tanh,
411  % asinh, acosh, atanh, log, log10, exp, pi, e
412  
413  % ---------- INPUT / OUTPUT ----------
414  % write('You saw me'), nl.
415  
416  % writeq('I show quotes'), nl.
417  
418  % You can read data with read
419  say_hi :-
420    write('What is your name? '),
421    read(X),
422    write('Hi '),
423    write(X).
424  
425  % say_hi.
426  % What is your name 'Derek'.
427  % Hi Derek
428  
429  fav_char :-
430    write('What is your favorite character? '),
431  
432    % Receives a char and saves its ascii value to X
433    get(X),
434    format('The Ascii value ~w is ', [X]),
435  
436    % Outputs Ascii value as the char
437    put(X),nl.
438  
439  % Write to a file by defining the file, text to write, connection
440  % to the file (Stream)
441  write_to_file(File, Text) :-
442    open(File, write, Stream),
443    write(Stream, Text), nl,
444    close(Stream).
445  
446  % Read from a file
447  read_file(File) :-
448          open(File, read, Stream),
449  
450          % Get char from the stream
451          get_char(Stream, Char1),
452  
453          % Outputs the characters until end_of_file
454          process_stream(Char1, Stream),
455          close(Stream).
456  
457  % Continue getting characters until end_of_file
458  % ! or cut is used to end backtracking or this execution
459  process_stream(end_of_file, _) :- !.
460  
461  process_stream(Char, Stream) :-
462          write(Char),
463          get_char(Stream, Char2),
464          process_stream(Char2, Stream).
465  
466  % ---------- HOW TO LOOP ----------
467  
468  % Use recursion to loop
469  count_to_10(10) :- write(10), nl.
470  
471  count_to_10(X) :-
472    write(X),nl,
473    Y is X + 1,
474    count_to_10(Y).
475  
476  % Receives Low (lowest value) and High (highest value)
477  count_down(Low, High) :-
478    % Assigns values between Low and High to Y
479    between(Low, High, Y),
480    % Assigns the difference to Z
481    Z is High - Y,
482    write(Z),nl,
483    % Continue looping until Y = 10
484    Y = 10.
485  
486  count_up(Low, High) :-
487    between(Low, High, Y),
488    Z is Y + Low,
489    write(Z), nl,
490    Y = 10.
491  
492  % Loop until they guess a number
493  % start is a dummy value used to start the looping
494  guess_num :- loop(start).
495  
496  % When they guess 15 they execute this message and exit
497  loop(15) :- write('You guessed it!').
498  
499  loop(X) :-
500    x \= 15,
501    write('Guess Number '),
502    read(Guess),
503    write(Guess),
504    write(' is not the number'), nl,
505    loop(Guess).
506  
507  % guess_num.
508  % Guess Number 12.
509  % 12 is not the number
510  % Guess Number 15.
511  % 15 is not the number
512  % You guessed it!
513  
514  % ---------- CHANGING THE DATABASE ----------
515  % Any predicate you plan to motify should be marked as dynamic before
516  % this predicate is used in any way
517  :- dynamic(father/2).
518  :- dynamic(likes/2).
519  :- dynamic(friend/2).
520  :- dynamic(stabs/3).
521  
522  father(lord_montague,romeo).
523  father(lord_capulet,juliet).
524  
525  likes(mercutio,dancing).
526  likes(benvolio,dancing).
527  likes(romeo,dancing).
528  likes(romeo,juliet).
529  likes(juliet,romeo).
530  likes(juliet,dancing).
531  
532  friend(romeo,mercutio).
533  friend(romeo,benvolio).
534  % friend(X, romeo) :- friend(romeo, X).
535  
536  stabs(tybalt,mercutio,sword).
537  stabs(romeo,tybalt,sword).
538  
539  % Add new clause to the database at the end of the list for the same
540  % predicate
541  % assertz(friend(benvolio, mercutio)).
542  % friend(benvolio, mercutio). = yes
543  
544  % Add clause at the start of the predicate list
545  % asserta(friend(mercutio, benvolio)).
546  % friend(mercutio, benvolio). = yes
547  
548  % Delete a clause
549  % retract(likes(mercutio,dancing)).
550  % likes(mercutio,dancing). = no
551  
552  % Delete all clauses that match
553  % retractall(father(_,_)).
554  % father(lord_montague,romeo). = no
555  
556  % Delete all matching clauses
557  % retractall(likes(_,dancing)).
558  % likes(_,dancing). = no
559  
560  % ---------- LISTS ----------
561  % You can store atoms, complex terms, variables, numbers and other
562  % lists in a list
563  % They are used to store data that has an unknown number of elements
564  
565  % We can add items to a list with the | (List Constructor)
566  % write([albert|[alice, bob]]), nl.
567  
568  % Get the length of a list
569  % length([1,2,3], X).
570  
571  % We can divide a list into its head and tail with |
572  % [H|T] = [a,b,c].
573  
574  % H = a
575  % T = [b,c]
576  
577  % We can get additional values by adding more variables to the left
578  % of |
579  
580  %[X1, X2, X3, X4|T] = [a,b,c,d].
581  
582  % We can use the anonymous variable _ when we need to reference a
583  % variable, but we don't want its value
584  % Let's get the second value in the list
585  % [_, X2, _, _|T] = [a,b,c,d].
586  
587  % We can use | to access values of lists in lists
588  % [_, _, [X|Y], _, Z|T] = [a, b, [c, d, e], f, g, h].
589  
590  % Find out if a value is in a list with member
591  % List1 = [a,b,c].
592  % member(a, List1). = yes
593  
594  % We could also get all members of a list with a variable
595  % member(X, [a, b, c, d]).
596  
597  % Reverse a list
598  % reverse([1,2,3,4,5], X).
599  
600  % Concatenate 2 lists
601  % append([1,2,3], [4,5,6], X).
602  
603  % Write items in list on separate line
604  write_list([]).
605  
606  write_list([Head|Tail]) :-
607    write(Head), nl,
608    write_list(Tail).
609  % write_list([1,2,3,4,5]). = Outputs the list
610  
611  % ---------- STRINGS ----------
612  % Convert a string into an Ascii character list
613  % name('A random string', X).
614  
615  % Convert a Ascii character list into a string
616  % name(X, [65,32,114,97,110,100,111,109,32,115,116,114,105,110,103]).
617  
618  % Append can join strings
619  join_str(Str1, Str2, Str3) :-
620  
621    % Convert strings into lists
622    name(Str1, StrList1),
623    name(Str2, StrList2),
624  
625    % Combine string lists into new string list
626    append(StrList1, StrList2, StrList3),
627  
628    % Convert list into a string
629    name(Str3, StrList3).
630  
631  % join_str('Another ', 'Random String', X). = X = 'Another Random String'
632  
633  % get the 1st char from a string
634  /*
635  name('Derek', List),
636  nth0(0, List, FChar),
637  put(FChar).
638  */
639  
640  % Get length of the string
641  atom_length('Derek',X).
\end{verbatim}

\section{Example}
\label{sec:org072957f}
\subsection{\texttt{epidemic.cpl}}
\label{sec:org7b19bb3}
\url{http://github.com/friguzzi/cplint/blob/master/prolog/examples/epidemic.cpl}

\begin{verbatim}
 1  /*
 2  Model of the development of an epidemic or a pandemic.
 3  From
 4  E. Bellodi and F. Riguzzi. Expectation Maximization over binary decision
 5  diagrams for probabilistic logic programs. Intelligent Data Analysis,
 6  17(2):343-363, 2013.
 7  */
 8  
 9  
10  epidemic : 0.6; pandemic : 0.3 :- flu(_), cold.
11  % if somebody has the flu and the climate is cold, there is the possibility
12  % that an epidemic arises with probability 0.6 and the possibility that a
13  % pandemic arises with probability 0.3
14  
15  cold : 0.7.
16  % it is cold with probability 0.7
17  
18  flu(david).
19  flu(robert).
20  % david and robert have the flu for sure
21  
22  /** <examples>
23  
24  ?- epidemic.  % what is the probability that an epidemic arises?
25  % expected result 0.588
26  ?- pandemic.  % what is the probability that a pandemic arises?
27  % expected result 0.357
28  
29  */
\end{verbatim}

\subsection{Load the library}
\label{sec:orgc1b1c2a}

\subsubsection{Step 1}
\label{sec:org58394c5}
Start swipl in the same directory where \texttt{epidemic.cpl} lives

\begin{verbatim}
swipl
\end{verbatim}

\subsubsection{Step 2}
\label{sec:org8abb6fc}
Then type into the \texttt{swipl} console "\texttt{[epidemic].}" and press enter.

\begin{verbatim}
[epidemic].
\end{verbatim}


This should load the \texttt{epidemic.cpl} program.

\subsubsection{Step 3}
\label{sec:org48e8316}

Then to calculate the probability of an epidemic, type the following into the console and press enter:

\begin{verbatim}
prob(epidemic,P).
\end{verbatim}

\subsubsection{Output}
\label{sec:orgad686a9}
\begin{verbatim}
 1  Welcome to SWI-Prolog (threaded, 64 bits, version 8.0.2)
 2  SWI-Prolog comes with ABSOLUTELY NO WARRANTY. This is free software.
 3  Please run ?- license. for legal details.
 4  
 5  For online help and background, visit http://www.swi-prolog.org
 6  For built-in help, use ?- help(Topic). or ?- apropos(Word).
 7  
 8  ?- [epidemic].
 9  true.
10  
11  ?- prob(epidemic,P).
12  P = 0.42 .
13  
14  ?-
\end{verbatim}

\section{cplint Glossary}
\label{sec:org279898b}

\begin{verbatim}
 1  cpl
 2  cplint
 3      [#prolog]
 4      [prolog package]
 5  
 6      cplint is a package for prolog that is
 7      used for probabilistic logic programming.
 8  
 9  prob/2
10      [#cplint]
11      [predicate]
12  
13      prob is a predicate for the cplint that
14      takes 2 arguments.
15  
16      Computes the probability of an atom.
17  
18          a:0.2:-
19              prob(b,P),
20              P > 0.2.
21  
22      Read about it:
23      - http://friguzzi.github.io/cplint/_build/html/index.html
\end{verbatim}
\end{document}
