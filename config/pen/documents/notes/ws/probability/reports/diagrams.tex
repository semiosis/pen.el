% Created 2024-04-15 Mon 17:11
% Intended LaTeX compiler: xelatex
\documentclass[11pt]{article}
\usepackage[mathletters]{ucs}
\usepackage{graphicx}
\usepackage{longtable}
\usepackage{wrapfig}
\usepackage{rotating}
\usepackage[normalem]{ulem}
\usepackage{amsmath}
\usepackage{amssymb}
\usepackage{capt-of}
\usepackage{hyperref}
\usepackage[margin=0.5in]{geometry}
\usepackage{fontspec}
\setmainfont{DejaVu Sans}
\author{Shane Mulligan}
\date{\textit{<2024-04-15 Mon>}}
\title{Chemistry diagrams}
\hypersetup{
 pdfauthor={Shane Mulligan},
 pdftitle={Chemistry diagrams},
 pdfkeywords={chemistry emacs},
 pdfsubject={},
 pdfcreator={Emacs 29.1.50 (Org mode 9.6.8)}, 
 pdflang={English}}
\begin{document}

\maketitle
\begin{verbatim}
1  
2       ··
3      :O :
4   ··  |   ··
5  :O =:S :-O :
6       ··  ··
7  
\end{verbatim}

Chemistry equations:
\begin{itemize}
\item Arrhenius equation
\item Butler–Volmer equation
\item Eyring equation
\item Henderson–Hasselbalch equation
\item Michaelis–Menten equation
\item Nernst equation
\item Schrödinger equation
\item Urey–Bigeleisen–Mayer equation
\end{itemize}

\section{Arrhenius equation}
\label{sec:orgca2303c}

\url{https://en.wikipedia.org/wiki/Arrhenius\_equation}

\begin{equation}
  {\displaystyle k=Ae^{\frac {-E_{\text{a}}}{RT}},}
\end{equation}

\section{Butler–Volmer equation}
\label{sec:org34cb50e}

\url{https://en.wikipedia.org/wiki/Butler\%E2\%80\%93Volmer\_equation}

\begin{equation}
  {\displaystyle j=j_{0}\cdot \left\{\exp \left[{\frac {\alpha _{\rm {a}}zF}{RT}}(E-E_{\rm {eq}})\right]-\exp \left[-{\frac {\alpha _{\rm {c}}zF}{RT}}(E-E_{\rm {eq}})\right]\right\}}
\end{equation}

\begin{itemize}
\item Eyring equation
\item Henderson–Hasselbalch equation
\item Michaelis–Menten equation
\item Nernst equation
\item Schrödinger equation
\item Urey–Bigeleisen–Mayer equation
\end{itemize}
\end{document}